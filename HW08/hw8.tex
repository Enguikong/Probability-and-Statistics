\documentclass{article}
\usepackage{xeCJK,amsmath,geometry,float,graphicx,amssymb,zhnumber,booktabs,setspace,tasks,verbatim,amsthm,amsfonts,mathdots}
\usepackage{listings,xcolor,diagbox}
\geometry{a4paper,scale=0.8}  
\renewcommand\arraystretch{2}
\title{概统作业 (Week 8)}
\author{PB20000113孔浩宇}
\begin{document}
\maketitle
\section{(P122 T48)}  %1
\begin{proof}
    \begin{enumerate}
        \item []
        \item [(1)]
        有$X$和$Y$边缘分布
        \[
            f(x,y) = 
            \frac{1}{2\pi \sqrt{1-\rho^2}} 
            e^{\{-\frac{1}{2(1-\rho^2)} (x^2 - 2\rho xy + y^2) \}}  
            \ \Rightarrow\ 
            X\sim N(0,1),\ Y\sim N(0,1)
        \]
        \[
            f_{X} (x) = \frac{1}{\sqrt{2\pi}} e^{-\frac{x^2}{2}},\ 
            f_{Y} (y) = \frac{1}{\sqrt{2\pi}} e^{-\frac{y^2}{2}}.
        \]
        有$Z$分布
        \[
            f_{Z} (z) = P(Z \leq z) 
            = \iint_{\frac{y-\rho x}{\sqrt{1-\rho^2}} \leq z} f(x,y) dx dy
            = \int_{-\infty}^{+\infty} dx \int_{-\infty}^{\sqrt{1-\rho^2}z + \rho x}dy f(x,y)
            = \frac{1}{\sqrt{2\pi}} e^{-\frac{z^2}{2}}  
        \]
        有$X,Z$联合分布
        \[
            f_{(X,Z)} (x,z) 
            = f(x, \sqrt{1-\rho^2}z + \rho x) \cdot |J|
            = \sqrt{1- \rho^2} \cdot f(x, \sqrt{1-\rho^2}z + \rho x)
            = \frac{1}{2\pi} e^{\{-\frac{1}{2} (x^2 + z^2) \}}  
            = f_{X}(x) \cdot f_{Z}(z) .
        \]
        即证.
        \item [(2)]
        \begin{align*}
            P(XY<0) 
            & = 1 - P(XY\geq 0)\\
            & = 1 - P(XY > 0)\\
            & = 1 - P(X>0, Y>0) - P(X<0, Y<0)
        \end{align*}
        由
        \[
            f(-x, -y) = f(x,y) 
            \ \Rightarrow\ 
            P(X<0,Y<0) = P(-x <0 , -y<0) = P(X>0, Y>0).    
        \]
        即
        \[
            P(XY<0) = 1 - 2P(X>0, Y>0).
        \]
        有
        \begin{align*}
            P(X>0, Y>0) 
            & = P(X>0, (\sqrt{1-\rho^2} Z + \rho X) > 0) \\  
            & = P(X>0 , (\sqrt{1-\rho^2} Z + \rho X) > 0) \\
            & = P(X>0 , Z > \frac{-\rho X}{\sqrt{1-\rho^2}}) \\
            & = \iint_{x>0, \sqrt{1-\rho^2} z + \rho x >0}  f_{(X,Z)}(x,z) dx dz.
        \end{align*}
        做代换
        \[
            \begin{cases}
                x = r \cos \alpha\\
                z = r \sin \alpha
            \end{cases} 
            \ \Rightarrow\ 
            |J|
            = \frac{\partial(x,z)}{\partial(r,\alpha)} 
            = 
            \begin{vmatrix}
                \cos \alpha & \sin \alpha\\
                -r\sin \alpha & -r \cos \alpha    
            \end{vmatrix}  
            = r.
            \ (0\leq \alpha <2\pi ,\ r>0)
        \]
        记$\rho = \cos \beta ,\ \sqrt{1- \rho^2} = \sin \beta \ (0<\beta <\frac{\pi}{2})$,有
        \[
            \iint_{x>0, \sqrt{1-\rho^2} z + \rho x >0}  f_{(X,Z)}(x,z) dx dz.
            = 
            \iint_{\sin \alpha >0, \cos (\alpha - \beta) > 0} r\cdot \frac{1}{2\pi} e^{-\frac{r^2}{2}}   dr d\theta .
        \]
        \begin{align*}
            P(X>0,Y>0)
            & = \iint_{\sin \alpha >0, \cos (\alpha - \beta) > 0} r\cdot \frac{1}{2\pi \cdot \sin \beta} e^{-\frac{r^2}{2}}   dr d\theta\\
            & = \int_{\sin \alpha >0, \cos (\alpha - \beta) > 0} \frac{1}{2\pi \cdot \sin \beta} d\theta \int_{0}^{+\infty}  -e^{-\frac{r^2}{2}} d(-\frac{r^2}{2})\\
            & = \int_{0}^{\frac{\pi}{2}} \frac{1}{2\pi} d\theta + \int_{\frac{3\pi}{2}+\beta}^{2\pi} \frac{1}{2\pi} d\theta \\
            & = \frac{\pi - \beta}{2\pi}\\
            & = \frac{1}{2} - \frac{\arccos \rho}{2\pi}
        \end{align*}
        故
        \[
            P(XY<0) = 1 - P(X>0, Y>0) = \pi^{-1} \arccos \rho .  
        \]
    \end{enumerate}
\end{proof}

\section{(P171 T1)}  %2
\begin{enumerate}
    \item [(0)]对$0<p<1$,记$q=1-p$,$\forall\ X\in \{4,\ 5,\ 6,\ 7 \}$,有
    \[
        P(X=n) = \binom{n-1}{n-4} \cdot (p^4 \cdot q^{n-4} + q^4 \cdot p^{n-4})   
    \]
    \[
        E(X) = 4\cdot P(X=4) + 5 \cdot P(X=5) + 6\cdot P(X=6) + 7\cdot P(X=7)
    \]
    \item [(1)]对$p=0.5$,有
    \[
        E(X) = 4 \cdot \frac{1}{8} + 5 \cdot 4 \cdot \frac{1}{16} + 
        6 \cdot 10 \cdot \frac{1}{32} + 7 \cdot 20 \cdot \frac{1}{64}
        = \frac{93}{16}
    \]
    \item [(1)]对$p=0.6$,有
    \[
        E(X) 
        = 
        4 \cdot \frac{3^4 + 2^4}{5^4}
        + 5 \cdot 4 \cdot \frac{3^4\cdot 2 + 2^4 \cdot 3}{5^5}
        + 6 \cdot 10 \cdot \frac{3^4\cdot 2^2 + 2^4 \cdot 3^2}{5^6} 
        + 7 \cdot 20 \cdot \frac{3^4\cdot 2^3 + 2^4 \cdot 3^3}{5^7}
        = \frac{17804}{3125}
    \]
\end{enumerate}


\section{(P171 T2)}  %3
\begin{proof}
    \begin{enumerate}
        \item []
        \item [(1)]
        \[
            E(X) 
            = \sum\limits_{k=1}^{\infty} k \cdot P(X=k)    
            = \sum\limits_{k=1}^{\infty} \sum\limits_{n=1}^{k} P(X=k)
            = \sum\limits_{n=1}^{\infty} \sum\limits_{k=n}^{\infty} P(X=k)
            = \sum\limits_{n=1}^{\infty} P(X\geq n)
        \]
        \item [(2)]
        \[
            E(X)
            = \int_{0}^{\infty} t f(t) dt 
            = \int_{0}^{\infty} f(t) dt \int_{0}^{t} dx
            = \int_{0}^{\infty} dx \int_{x}^{\infty} f(t) dt
            = \int_{0}^{\infty} (1- F(x)) dx.
         \]
        \item [(3)]以$I(A)$表示事件$A$的示性函数,则有
        \[
            E(X) 
            = E\left( \int_{0}^{X} dx \right)
            = E\left( \int_{0}^{\infty} I(X>x)dx \right)
            = \int_{0}^{\infty} E\left[ I(X>x) \right]
            = \int_{0}^{\infty} P(X>x) dx
        \]
        又$P(X>x) = 1 -F(x)$,即证
        \[
            E(X) = \int_{0}^{\infty} (1- F(x)) dx.
        \]
    \end{enumerate}
\end{proof}

\section{(P171 T8)}  %4
\begin{enumerate}
    \item [(1)]
    用$E(T_i)$表示从第$i-1$种到第$i$种需要买的卡片的期望,则单次买到第$i$种卡片的概率为$p=\displaystyle{\frac{n}{n-i+1}}$.
    记$q = 1 - p$.
    \[
        E(T_i) 
        = \sum\limits_{k=1}^{\infty} k\cdot q^{k-1}\cdot p
        = p \cdot \left(\sum\limits_{k=1}^{\infty} q^{k} \right)'
        = \frac{1}{p}
        = \frac{n}{n-i+1}.
    \]
    \[
        E(X_n)
        = \sum\limits_{i=1}^{n} E(T_i)
        = \sum\limits_{i=1}^{n} \frac{n}{n-i+1}
        = n \cdot \sum\limits_{i=1}^{n} \frac{1}{i}
        = 12 \cdot \sum\limits_{i=1}^{12} \frac{1}{i}
        = \frac{86021}{2310}
        \approx 37.24
    \]
    \item [(2)]
    \[
        \lim\limits_{n\to \infty} E\left(\frac{X_n}{n \ln n} \right)
        = \lim\limits_{n\to \infty} \frac{E(X_n)}{n \ln n}   
        = \lim\limits_{n\to \infty} \frac{n \cdot \sum\limits_{i=1}^{n} \frac{1}{i} }{n \ln n}   
        = \lim\limits_{n\to \infty} \frac{\sum\limits_{i=1}^{n} \frac{1}{i} }{\ln n}   
    \]
    记$x_n = \sum\limits_{i=1}^{n} \frac{1}{i} - \ln n$,有
    \[
        x_n 
        \geq \sum\limits_{i=1}^{n} \ln (1+\frac{1}{i}) - \ln n
        = \ln (n+1) - \ln n
        \geq 0.
    \]
    \[
        x_{n+1} - x_n 
        = \frac{1}{n+1} - \ln \frac{n+1}{n}
        = \frac{1}{n+1} - \ln \left(1+\frac{1}{n}\right)
        \leq \frac{1}{n+1} - \frac{1}{n+1}
        = 0.
    \]
    由$\{x_n\}$单调有界可知$\lim\limits_{n\to \infty} x_n$存在,记为$x$.则
    \[
        \lim\limits_{n\to \infty} \frac{\sum\limits_{i=1}^{n} \frac{1}{i} }{\ln n}   
        =
        \lim\limits_{n\to \infty} \frac{\ln n + x}{\ln n}
        =
        \lim\limits_{n\to \infty} \left(1 + \frac{x}{\ln n}\right)
        =
        1.
    \]
    即
    \[
        E\left(\frac{X_n}{n \ln n}\right) = 1.    
    \]
\end{enumerate}

\section{(P173 T11)}  %5
\begin{proof}
    不妨记$P(X=x_i) = p_i$,其中$p_i >0\ (i=1,2,\ldots,k),\ \sum\limits_{i=1}^{k} p_i =1$,不妨记$\max\limits_{1\leq i \leq k} x_i = x_m$,有
    \[
        E[X^n] = \sum\limits_{i=1}^{k} x_i^n p_i
        \ \Rightarrow\ 
        \lim\limits_{n\to \infty} \frac{E[X^n]}{x_m^n} = p_m .
    \]
    故
    \[
        \lim\limits_{n\to \infty} \frac{E[X^{n+1}]}{E[X^n]} 
        =
        \lim\limits_{n\to \infty} \frac{E[X^{n+1}]}{x_m^{n+1}} \frac{x_m^n \cdot x_m}{E[X^n]} 
        =
        x_m
        =
        \max\limits_{1\leq i \leq k} x_i .
    \]
\end{proof}

\end{document}