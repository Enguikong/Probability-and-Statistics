\documentclass{article}
\usepackage{xeCJK,amsmath,geometry,float,graphicx,amssymb,zhnumber,booktabs,setspace,tasks,verbatim,amsthm,amsfonts,mathdots}
\usepackage{listings,xcolor,diagbox}
\geometry{a4paper,scale=0.8}  
\renewcommand\arraystretch{2}
\title{概统作业 (Week 4)}
\author{PB20000113孔浩宇}
\begin{document}
\maketitle
\section{(P82 第12题)}  %1
\begin{enumerate}
    \item [(1)]记一次产卵数量为$X$.
    \[
        P(Y=k) 
        =
        \sum\limits_{n=k}^{+\infty} P(X=n , Y=k) 
        =
        \sum\limits_{n=k}^{+\infty} P(X=n) P(Y=k \mid X=n)
    \]
    又
    \[
        P(X=n) P(Y=k \mid X=n)
        =
        \displaystyle{\frac{\lambda^n}{n!}} e^{-\lambda}\cdot
        \binom{n}{k} \cdot p^{k} {(1-p)}^{n-k}
        \quad (n\geq k)
    \]
    故
    \begin{align*}
        P(Y=k)
        &=
        \sum\limits_{n=k}^{+\infty} 
        \displaystyle{\frac{\lambda^n}{n!}} e^{-\lambda}\cdot
        \displaystyle{\frac{n!\cdot p^{k}}{k!\cdot {(n-k)!}}} \cdot  {(1-p)}^{n-k} \\
        &=
        \displaystyle{\frac{\lambda^k \cdot p^{k}}{k!}} e^{-\lambda}\cdot
        \sum\limits_{n=k}^{+\infty} 
        \displaystyle{\frac{{(\lambda -\lambda p)}^{n-k}}{{(n-k)!}}}  \\
        &=
        \displaystyle{\frac{\lambda^k \cdot p^{k}}{k!}} e^{-\lambda}\cdot
        e^{\lambda - \lambda p}\\
        &=
        \displaystyle{\frac{\lambda^k \cdot p^{k}}{k!}} e^{-\lambda p} .
    \end{align*}
    同理有
    \[
        P(Z=k)
        =
        \displaystyle{\frac{\lambda^k \cdot {(1-p)}^{k}}{k!}} e^{-\lambda (1-p)} .
    \]
    即$Y$服从参数为$\lambda p$的泊松分布,$Z$服从参数为$\lambda (1-p)$的泊松分布.
    \item [(2)]
    \[
        P(Y=m,Z=n) 
        =
        \displaystyle{\frac{\lambda^{m+n}}{(m+n)!}} e^{-\lambda}\cdot
        \frac{{(m+n)!}}{m!\cdot n!} p^{m} {(1-p)}^{n}
        =
        \frac{\lambda^{m+n}\cdot p^{m} {(1-p)}^{n} }{m!\cdot n!}\cdot e^{-\lambda} .
    \]
    \[
        P(Y=m) P(Z=n)
        =
        \frac{\lambda^m \cdot p^{m}}{m!} e^{-\lambda p} 
        \cdot 
        \frac{\lambda^n \cdot {(1-p)}^{n}}{n!} e^{-\lambda (1-p)}
        =
        \frac{\lambda^{m+n}\cdot p^{m} {(1-p)}^{n} }{m!\cdot n!}\cdot e^{-\lambda} .
    \]
    $P(Y=m,Z=n)=P(Y=m)P(Z=n)$,故$Y,Z$相互独立.
\end{enumerate}

\section{(P83 第19题)}  %2
\begin{enumerate}
    \item []由$F(x)$的右连续性有
    \[
        \lim\limits_{x\to -1+} F(x) = F(-1)
        \ \Rightarrow\ 
        -a+b=\frac{1}{8} .    
    \]
    \item []由于
    \[
        F(1) = P(X\leq 1) = P(X<1) + P(X=1)    
    \]
    且$P(X<1) =\lim\limits_{x\to 1-} F(x)=a+b$,可得
    \[
        a+b+\frac{1}{4}=1    
    \]
    联立以上方程解得
    \[
        a=\frac{5}{16},\ b=\frac{7}{16}.    
    \]
\end{enumerate}

\section{(P83 第21题)}  %3
\begin{enumerate}
    \item [(1)]
    \[
        \int_{-\infty}^{+\infty} \frac{a}{1+x^2} \,dx =1 
        \ \Leftrightarrow\
        a\cdot \arctan x \mid _{-\infty}^{+\infty} =1
        \ \Leftrightarrow\
        a\cdot (\frac{\pi}{2} - (-\frac{\pi}{2})) =1
        \ \Leftrightarrow\ 
        a=\frac{1}{\pi}.
    \]
    即$a=\displaystyle{\frac{1}{\pi}}$.
    \item [(2)]
    \[
        F(x)= \int_{-\infty}^{x} \frac{1}{\pi(1+t^2)} \,dt =
        \frac{1}{\pi}(\arctan(x)+\frac{\pi}{2}).
    \]
    \item [(3)]
    \[
        P(|X|<1)=P(-1<X<1)
        =F(1)-F(-1)
        =\frac{1}{\pi}[\arctan(1)- \arctan(-1)]
        =\frac{1}{2}.  
    \]
\end{enumerate}

\section{(P84 第26题)}  %4
\begin{enumerate}
    \item []单次观测到$x>2$的概率为
    \[
        p=P(X>2)=\displaystyle{\frac{4-2}{4-1}=\frac{2}{3}}.    
    \]
    \item []设三次独立观测中观测值大于2的次数为$Y$,则
    \[
        P(Y\geq 2)
        =P(Y=2)+P(Y=3)
        =3\cdot p^2 (1-p) + p^{3}
        =\frac{20}{27} .
    \]
    即三次独立观测中至少两次观测值大于2的概率为$\displaystyle{\frac{20}{27}}$.
\end{enumerate}

\section{(P84 第28题)}  %5
\begin{enumerate}
    \item [(1)]设检修时间为$X$,$f(x)=e^{-X}\ (X>0)$.
    \[
        P(X> 2)
        =1-P(X\leq 2)
        =1- F(2)  
        =e^{-2}.
    \]
    \item [(2)]
    \[
        P(X>4|X>2)=P(X>2)=e^{-2}.
    \]
\end{enumerate}

\section{(P84 第29题)}  %6
\begin{enumerate}
    \item []单次等待时间超过10min的概率为
    \[
        p = P(X> 10) = 1-F(10) = e^{-2}.
    \]
    \item []设一个月内未接受服务而离开的次数为$Y$,则
    \[
        P(Y\geq 1) = 1- P(Y=0) = 1 - {(1-p)}^{5}
        =1 - {(1-e^{-2})}^{5}
        \approx 0.5167. 
    \]
\end{enumerate}

\end{document}