\documentclass{article}
\usepackage{xeCJK,amsmath,geometry,float,graphicx,amssymb,zhnumber,booktabs,setspace,tasks,verbatim,amsthm,amsfonts,mathdots}
\usepackage{listings,xcolor,diagbox}
\usepackage{unicode-math}
\usepackage{media9}
\usepackage{multimedia}
\geometry{a4paper,scale=0.8}  
\renewcommand\arraystretch{2}
\title{概统作业 (Week 15)}
\author{PB20000113孔浩宇}
\begin{document}
\maketitle
\section{}  %1
此检验犯第二类错误的概率为
\[
    \alpha_{2\Psi} (\theta)
    = 1 - \beta_{\Psi} (\theta)
    = 1 - P_{\theta = \theta_1} \left\{ X_1 = X_2 = X_3 = 1 \right\}    
    = 1 - \theta_1^6 .
\]

\section{}  %2
\begin{enumerate}
    \item [(1)]有最大似然函数
    \[
        L(X_1, X_2, \ldots, X_{n} ; p)
        = \prod\limits_{i=1}^{n} p {(1-p)}^{k-1}
        % = p^{130} {(1-p)}^{31 + 2*20 + 3*9 + 4*6 + 5*5 + 6*4 + 7*2 + 8*1 + 9*1 + 10*2 + 11*1}    
        = p^{130} {(1-p)}^{233}
    \]
    \[
        \ln L
        = 130\cdot \ln p + 233\cdot \ln (1-p)    
    \]
    令
    \[
        \frac{\partial \ln L}{\partial p}
        = \frac{130}{p} + \frac{233}{1-p} 
        = 0.   
    \]
    得
    \[
        \widehat{p} = -\frac{130}{103} \approx -1.262   
    \]
    检验二阶导
    \[
        \frac{\partial^2 \ln L}{\partial p^2}
        = -\frac{130}{p^2} + \frac{233}{{(1-p)}^2} 
    \]
    当$p = \widehat{p}$时,二阶偏导为负,故所求驻点为$L$极大值点,最大似然估计为
    \[
        \widehat{p} = -\frac{130}{103} \approx -1.262  
    \]
    \item [(2)]
\end{enumerate}

\section{}  %3
记一班

\end{document}