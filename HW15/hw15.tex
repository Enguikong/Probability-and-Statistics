\documentclass{article}
\usepackage{xeCJK,amsmath,geometry,float,graphicx,amssymb,zhnumber,booktabs,setspace,tasks,verbatim,amsthm,amsfonts,mathdots}
\usepackage{listings,xcolor,diagbox}
\usepackage{unicode-math}
\usepackage{media9}
\usepackage{multimedia}
\geometry{a4paper,scale=0.8}  
\renewcommand\arraystretch{2}
\title{概统作业 (Week 15)}
\author{PB20000113孔浩宇}
\begin{document}
\maketitle
\section{}  %1
此检验犯第二类错误的概率为
\[
    \alpha_{2\Psi} (\theta)
    = 1 - \beta_{\Psi} (\theta)
    = 1 - P_{\theta = \theta_1} \left\{ X_1 = X_2 = X_3 = 1 \right\}    
    = 1 - \theta_1^6 .
\]

\section{}  %2
\begin{enumerate}
    \item [(1)]有最大似然函数
    \[
        L(X_1, X_2, \ldots, X_{n} ; p)
        = \prod\limits_{i=1}^{n} p {(1-p)}^{k-1}
        % = p^{130} {(1-p)}^{31 + 2*20 + 3*9 + 4*6 + 5*5 + 6*4 + 7*2 + 8*1 + 9*1 + 10*2 + 11*1}    
        = p^{130} {(1-p)}^{233}
    \]
    \[
        \ln L
        = 130\cdot \ln p + 233\cdot \ln (1-p)    
    \]
    令
    \[
        \frac{\partial \ln L}{\partial p}
        = \frac{130}{p} + \frac{233}{p-1} 
        = 0.   
    \]
    得
    \[
        \widehat{p} = \frac{130}{363} \approx 0.358 
    \]
    检验二阶导
    \[
        \frac{\partial^2 \ln L}{\partial p^2}
        = -\frac{130}{p^2} - \frac{233}{{(p-1)}^2} 
    \]
    当$p = \widehat{p}$时,二阶偏导为负,故所求驻点为$L$极大值点,最大似然估计为
    \[
        \widehat{p} = \frac{130}{363} \approx 0.358   
    \]
    \item [(2)]
    \[
        \begin{tabular}{c|ccccccc}
            类别 & 1 & 2 & 3 & 4 & 5 & 6 & $\geq 7$ \\
            \hline
            $\widehat{E}$ & 46.54 & 29.879 & 19.182 & 12.315 & 7.906 & 5.076 & 9.102 \\
            $O$ & 48 & 31 & 20 & 9 & 6 & 5 & 11\\
            $O - \widehat{E}$ & 1.46 & 1.121 & 0.818 & -3.315 & -1.906 & -0.076 & 1.898 
        \end{tabular}
    \]
    有检验
    \[
        H_0 : P(X = k) = p\cdot {(1-p)}^{k-1}\ (k = 1,2,\ldots)
        \quad \leftrightarrow \quad
        H_1 : \exists\ k\in \mathbb{N}^* , P(X = k) \neq p \cdot {(1-p)}^{k-1} 
    \]
    构造统计量
    \[
        Z 
        = \sum \frac{{(O - \widehat{E})}^2}{\widehat{E}}
        \sim \chi_{k-r-2}^{2}
        \qquad (n\to \infty)
    \]
    有拒绝域
    \[
        W = 
        \left\{
            Z > \chi_{k-r-1}^{2} (\alpha)
        \right\}    
    \]
    代入数据$k = 7, r = 1, \alpha = 0.05$,得
    \[
        \chi_{k-r-1}^{2} (\alpha)
        = \chi_{7}^{2} (0.05)    
        = 14.067
    \]
    \[
        Z 
        = \sum \frac{{(O - \widehat{E})}^2}{\widehat{E}}
        = 1.8715
        < 14.067
        \ \Rightarrow\ 
        Z \notin W
    \]
    故不能拒绝$H_0$,即在显著性水平$\alpha = 0.05$下认为$X$服从几何分布。
\end{enumerate}

\section{}  %3
有检验
\[
    H_0 : p_{ij} = P(X = i) P(Y = j) \ (i=1,\ldots,a;\ j = 1,\ldots,b)
    \quad \leftrightarrow \quad
    H_1 : \exists\ i,j\in \mathbb{N}^* , p_{ij} \neq P(X = i) P(Y = j)
\]
取检验统计量
\[
    Z 
    = \sum\limits_{i=1}^{a} \sum\limits_{j=1}^{b} \frac{(n\cdot n_{ij} - n_{i.} \cdot n_{.j})}{n\cdot n_{i.} \cdot n_{.j}}    
    \sim \chi_{(a-1)(b-1)}^{2}
    \qquad (n\to \infty)
\]
拒绝域
\[
    W = 
    \left\{
        Z > \chi_{(a-1)(b-1)}^2 (\alpha)
    \right\}    
\]
代入数据$a = 6,\ b = 2,\ \alpha = 0.05$,得
\[
    \chi^2_{(a-1)(b-1)} (\alpha)
    = \chi^2_{6} (0.05)
    = 12.592
\]
\[
    Z 
    = \sum\limits_{i=1}^{a} \sum\limits_{j=1}^{b} \frac{(n\cdot n_{ij} - n_{i.} \cdot n_{.j})}{n\cdot n_{i.} \cdot n_{.j}}    
    = 3.1922
    < 12.592
    \ \Rightarrow\ 
    Z \notin W
\]
故不能拒绝$H_0$,即在显著性水平$\alpha = 0.05$下认为两个班级的英语水平大致相等。

\end{document}