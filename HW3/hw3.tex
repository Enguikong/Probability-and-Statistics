\documentclass{article}
\usepackage{xeCJK,amsmath,geometry,float,graphicx,amssymb,zhnumber,booktabs,setspace,tasks,verbatim,amsthm,amsfonts,mathdots}
\usepackage{listings,xcolor,diagbox}
\geometry{a4paper,scale=0.8}  
\renewcommand\arraystretch{2}
\title{概统作业 (Week 3)}
\author{PB20000113孔浩宇}
\begin{document}
\maketitle
\section{(P45 第37题)}  %1
\begin{enumerate}
    \item [(1)]
    \[
        P(A)=
        P(AC)+P(A\overline{C})
        =P(A|C)\cdot P(C) + P(A\overline{C})\cdot P(\overline{C})
        =0.55 .
    \]
    \item [(2)]
    \[
        P(B)=  
        P(BC)+P(B\overline{C})
        =P(B|C)\cdot P(C) + P(B\overline{C})\cdot P(\overline{C})
        =0.5 .
    \]
    \item [(3)]
    \begin{align*}
        P(AB)
        & =
        P(ABC) + P(AB\overline{C})\\
        & =
        P(AB|C)\cdot P(C) + P(AB\overline{AB|\overline{C}})\cdot P(\overline{C})\\
        & =
        P(A|C)\cdot P(B|C)\cdot P(C) + P(A|\overline{C})\cdot P(B|\overline{C})\cdot P(\overline{C})\\\
        & =
        0.9\times 0.9\times 0.5 + 0.2\times 0.2 \times 0.5\\
        & =
        0.425 
    \end{align*}
    \item [(4)]
    \begin{proof}
        \[
            P(A)P(B)=
            0.55\times 0.5 
            =0.275
            \neq P(AB).    
        \]
    \end{proof}
\end{enumerate}

\section{(P45 第38题)}  %2
\begin{enumerate}
    \item []设第一次射中为事件$A$,第二次射中为事件$B$,第三次射中为事件$C$.
    \item [(1)]设恰有一次射中为事件$M$
    \begin{align*}
        P(M)
        & = 
        P(A \overline{B} \overline{C})+ P(\overline{A} B \overline{C})+ P(\overline{A}\overline{B}C)\\
        & =
        0.5\times 0.4\times 0.2 + 0.5\times 0.6\times 0.2 + 0.5\times 0.4\times 0.8\\
        & =
        0.26.
    \end{align*}
    \item [(2)]设至少有一次射中为事件$N$
    \[
        P(N) = 1- P(\overline{A} \overline{B} \overline{C})
        = 0.96.
    \]
\end{enumerate}

\section{(P81 第4题)}  %3
\begin{enumerate}
    \item []设营收为$X$万元,由题意可得$X$取值有$10$,$5$,$0$,$-2$,一天内发生故障的概率$p=0.2$.
    \[
        P(X=10)
        =\binom{5}{0} {p}^{0} {(1-p)}^{5} 
        =\frac{1024}{3125}.
    \]    
    \[
        P(X=5)
        =\binom{5}{1} {p}^{1} {(1-p)}^{4} 
        =\frac{256}{625}.   
    \]
    \[
        P(X=0)
        =\binom{5}{2} {p}^{2} {(1-p)}^{3} 
        =\frac{128}{625}   
    \]
    \[
        P(X=-2)
        =1-P(X=10)-P(X=5)-P(X=0)  
        =\frac{181}{3125}.  
    \]    
    \item []分布律如图
    \begin{table}[!ht]
        \centering
        \begin{tabular}{c|cccc}
            X & 10 & 5 & 0 & -2 \\ \hline
            P & $\displaystyle{\frac{1024}{3125}}$ & $\displaystyle{\frac{256}{625}}$ & $\displaystyle{\frac{128}{625}}$ & $\displaystyle{\frac{181}{3125}}$ 
        \end{tabular}
    \end{table}
\end{enumerate}

\section{(P81 第8题)}  %4
\begin{enumerate}
    \item [(1)]
    \[
        P(X\geq 1)=1-P(X=1)-P(X=0)
        =1-20 \cdot \binom{20}{1}    
    \]
    \item [(2)]
\end{enumerate}

\section{(P81 第9题)}  %5



\end{document}