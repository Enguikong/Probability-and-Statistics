\documentclass{article}
\usepackage{xeCJK,amsmath,geometry,float,graphicx,amssymb,zhnumber,booktabs,setspace,tasks,verbatim,amsthm,amsfonts,mathdots}
\usepackage{listings,xcolor,diagbox}
\usepackage{unicode-math}
\usepackage{media9}
\usepackage{multimedia}
\geometry{a4paper,scale=0.8}  
\renewcommand\arraystretch{2}
\title{概统作业 (Week 14)}
\author{PB20000113孔浩宇}
\begin{document}
\maketitle
\section{(P306 T6)}  %1
\begin{enumerate}
    \item [(1)]对$X_{(n)}$有分布函数和密度函数
    \[
        F_{n} (x) = 
        \begin{cases}
            0 , & (x \leq 0)\\
            \\
            \displaystyle{\frac{x^n}{\theta^n}} , & (x\in (0,\theta))\\
            \\
            1 . & (x\geq \theta)
        \end{cases} 
        \qquad
        f_{n} (x) = 
        \begin{cases}
            \displaystyle{\frac{n\cdot x^{n-1}}{\theta^{n}}} , & (x\in (0 , \theta))\\
            \\
            0. & (\mbox{其他})
        \end{cases}    
    \]
    故功效函数为
    \[
        \beta_{\Psi }(\theta)
        = \mathbb{P}_{\theta} \left( X_{(n)} \leq 2.5 \big\vert H_{0} \right)  
        = F_{n} (2.5) 
        = {\left(\frac{2.5}{\theta}\right)}^{n}
        \quad (\theta \geq 3)
    \]
    对于$\forall\ \theta \in H_{0}$,有
    \[
        \theta \geq 3,\ 
        \alpha_{1 \Psi} (\theta) 
        = \beta_{\Psi} (\theta)
        = {\left(\frac{2.5}{\theta}\right)}^{n} 
        \leq {\left(\frac{2.5}{3}\right)}^{n} 
    \]
    故有显著性水平
    \[
        \alpha 
        = {\left(\frac{2.5}{3}\right)}^{n}
        = {\left(\frac{5}{6}\right)}^{n}
    \]
    \item [(2)]
    \[
        \alpha \leq 0.05
        \ \Rightarrow\ 
        {\left(\frac{5}{6}\right)}^{n} \leq 0.05
        \ \Rightarrow\ 
        n \geq \log_{\frac{5}{6}} 0.05
        \ \Rightarrow\ 
        n \geq 16.43
    \]
    即$n$至少为$17$.
\end{enumerate}

\section{(P307 T9)}  %2
    检验
    \[
        H_0 : \mu = 105.02 
        \quad \leftrightarrow \quad
        H_1 : \mu \neq 105.02        
    \]
    8个省的数据分别记为$X_1 , X_2, \ldots, X_8$,记$n = 8$,有统计量
    \[
        \overline{X} 
        % = \frac{109.45 + 104.32 + 100.45 + 104.90 + 103.99 + 110.47 + 100.60 + 106.16}{8}
        = \frac{1}{n} \sum\limits_{i=1}^{n} X_i
        = 105.0425
    \]
    \[
        S^2
        = \frac{1}{n-1} \sum\limits_{i=1}^{n} {(X_i - \overline{X})}^2
        = 13.23
    \]
    取检验统计量
    \[
        T 
        = \frac{\sqrt{n} (\overline{X} - \mu)}{S}  
        \sim t_{n-1}
        \quad (\mu = \mu_{0})
    \]
    拒绝域
    \[
        W =
        \left\{
            |T| > t_{n-1}\left(\frac{\alpha}{2}\right)     
        \right\}
    \]
    代入数据,得
    \[
        T 
        = \frac{\sqrt{8}\times 0.0225}{\sqrt{13.23}} 
        = 0.0175,
        \qquad
        t_{n-1}\left(\frac{\alpha}{2}\right)
        = t_{7}(0.025)
        = 2.3646
    \]
    有
    \[
        |T| = 0.0175 < 2.3646 = t_{n-1}\left(\frac{\alpha}{2}\right)    
    \]
    不在拒绝域内,故在显著性水平$5\%$下,不能拒绝$H_0$.

\section{}  %3
\begin{enumerate}
    \item [(1)]不妨记旧工艺下样本分布为$X\sim N(\mu_1, \sigma_1^2)$,$m=12$,
    有样本均值$\overline{X}$与样本方差$S_1^2$
    \[
        \overline{X}_1
        = \frac{6 + 4 + 5 + 5 + 6 + 5 + 5 + 6 + 4 + 6 + 7 + 4}{12}    
        = 5.25
    \]
    \[
        S_1^2
        = \frac{1}{m-1} \sum\limits_{i=1}^{m} {(X_i - \overline{X})}^2
        = 0.9318
    \]
    记新工艺下样本分布为$X\sim N(\mu_2, \sigma_2^2)$,$n = 12$,
    有样本均值$\overline{Y}$与样本方差$S_2^2$
    \[
        \overline{Y}
        = \frac{2 + 1 + 2 + 2 + 1 + 0 + 3 + 2 + 1 + 0 + 1 + 3}{12}
        = 1.5    
    \]
    \[
        S_2^2
        = \frac{1}{n-1} \sum\limits_{i=1}^{n} {(Y_i - \overline{Y})}^2
        = 1
    \]
    取检验统计量
    \[
        F = \frac{S_1^2}{S_2^2}  \sim F(m-1,n-1) \quad (\sigma_1^2 = \sigma_2^2)
    \]
    检验
    \[
        H_0 : \sigma_1^2 = \sigma_2^2 
        \quad \leftrightarrow \quad
        H_1 : \sigma_1^2 \neq \sigma_2^2 
    \]
    拒绝域
    \[
        W = 
        \left\{
            F < F_{11,11} \left(1 - \frac{\alpha}{2}\right)
            \ \mbox{或}\ 
            F > F_{11,11} \left(\frac{\alpha}{2}\right)
        \right\}
    \]
    代入数据,有
    \[
        F_{11,11} \left(1 - \frac{\alpha}{2}\right)
        = F_{11,11} (0.975)
        = 0.288
        \qquad
        F_{11,11} \left(\frac{\alpha}{2}\right)
        = F_{11,11} (0.025)
        = 3.474
    \]
    \[
        F 
        = \frac{S_1^2}{S_2^2}
        = 0.9318 \notin W 
    \]
    故不能拒绝$H_0$,认为两个总体的方差想等。
    \item [(2)]即检验
    \[
        H_0 : \mu_1 - \mu_2 \geq 3
        \quad \leftrightarrow \quad
        H_1 : \mu_1 - \mu_2 <3 
    \]
    取检验统计量
    \[
        T = \frac{\overline{X} - \overline{Y} - 3}
        {S_T \sqrt{\frac{1}{m} + \frac{1}{n}}}    
        \sim t_{m+n-2}
        \quad (\mu_1 - \mu_2 = 3)
    \]
    其中有优良点估计
    \[
        S_T = \sqrt{\frac{(m-1) S_1^2 + (n-1) S_2^2}{m+n-2}}    
    \]
    拒绝域
    \[
        W = 
        \left\{
            T < - t_{m+n-2} (\alpha)
        \right\}    
    \]
    代入数据,得
    \[
        S_T 
        = \sqrt{\frac{(m-1) S_1^2 + (n-1) S_2^2}{m+n-2}}  
        = \sqrt{\frac{11\times 0.9318 + 11\times 1}{12+12-2}}
        = 0.9828   
    \]
    \[
        T
        = \frac{\overline{X} - \overline{Y} - 3}{S_T \sqrt{\frac{1}{m} + \frac{1}{n}}}     
        = \frac{5.25 - 1 - 3}{0.9828 \times \sqrt{\frac{1}{12} + \frac{1}{12}}}
        = 1.8693
    \]
    \[
        t_{m+n-2}(\alpha) 
        = t_{22}(0.05)
        = 1.7171 
    \]
    有
    \[
        T = 1.8693 > -1.7171
        \ \Rightarrow\ 
        T \notin W    
    \]
    故不能拒绝$H_0$,可以认为旧工艺下NDMA平均含量比新工艺下显著地大3.
\end{enumerate}

\section{(P312 T38)}  %4
    记马克·吐温文中3个字母组成的单字的比例为$X\sim N(\mu_1, \sigma_1^2)$,
    $m = 8$,有统计量
    \[
        \overline{X} = \frac{1}{m} \sum\limits_{i=1}^{m} X_i = 0.2319
    \]
    \[
        S_1^2 
        = \frac{1}{m-1} \sum\limits_{i=1}^{m} {\left(X_i - \overline{X}\right)}^2
        = 2.12 \times 10^{-4}
    \]
    记斯诺德格拉斯文中3个字母组成的单字的比例为$Y\sim N(\mu_2, \sigma_2^2)$,
    $n = 10$,有统计量
    \[
        \overline{Y} = \frac{1}{n} \sum\limits_{i=1}^{n} Y_i = 0.2097
    \]
    \[
        S_2^2 
        = \frac{1}{n-1} \sum\limits_{i=1}^{n} {\left(Y_i - \overline{Y}\right)}^2
        = 9.33 \times 10^{-5}
    \]
    有
    \[
        \sigma_1^2 = \sigma_2^2 = \sigma^2 \quad (\sigma^2\ \mbox{未知})  
    \]
    检验
    \[
        H_0 : \mu_1 = \mu_2 
        \quad \leftrightarrow \quad
        H_1 : \mu_1 \neq \mu_2         
    \]
    取检验统计量
    \[
        T = \frac{\overline{X} - \overline{Y}}
        {S_T \sqrt{\frac{1}{m} + \frac{1}{n}}}    
        \sim t_{m+n-2}
        \quad (\mu_1 - \mu_2 = 3)
    \]
    其中有优良点估计
    \[
        S_T = \sqrt{\frac{(m-1) S_1^2 + (n-1) S_2^2}{m+n-2}}    
    \]
    拒绝域
    \[
        W = 
        \left\{
            |T| > t_{m+n-2} (\alpha / 2)
        \right\}    
    \]
    代入数据,得
    \[
        S_T 
        = \sqrt{\frac{(m-1) S_1^2 + (n-1) S_2^2}{m+n-2}}  
        = \sqrt{\frac{7\times 2.12\times 10^{-4} + 9\times 9.33\times 10^{-5}}{8+10-2}}
        = 0.01205
    \]
    \[
        T
        = \frac{\overline{X} - \overline{Y}}{S_T \sqrt{\frac{1}{m} + \frac{1}{n}}}     
        = \frac{0.2319 - 0.2097}{0.01205 \times \sqrt{\frac{1}{8} + \frac{1}{10}}}
        = 3.884
    \]
    \[
        t_{m+n-2}(\alpha /2 ) 
        = t_{16}(0.025)
        = 2.1199
    \]
    有
    \[
        |T| = 3.884 > 2.1199
        \ \Rightarrow\ 
        T \in W    
    \]
    故拒绝$H_0$,即在显著性水平$\alpha = 0.05$下有差异,不能认为是一个人。

\section{A}  %5
\begin{proof}
    \begin{enumerate}
        \item []
        \item []对参数$\mu$ 进行检验
        \[
            H_0 : \mu = \mu_0
            \quad \leftrightarrow \quad
            H_1 : \mu \neq \mu_0   
        \]
        不妨记检验统计量为$T_n$.
        \item [(1)]$\sigma^2$已知,此时拒绝域
        \[
            \left\{
                |Z| > u_{\alpha / 2}
            \right\}    
        \]
        由$\alpha = 0.05$时接受$H_0$,有
        \[
            |T_n| \leq u_{0.025}    
        \]
        又
        \[
            u_{0.005} > u_{0.025}
            \ \Rightarrow\ 
            |T_n| < u_{0.005}   
        \]
        即此时$\alpha = 0.01$时仍接受$H_0$。

        \item [(2)]$\sigma^2$未知,此时拒绝域
        \[
            \left\{
                |T| > t_{n-1} (\alpha / 2)
            \right\}    
        \]
        由$\alpha = 0.05$时接受$H_0$,有
        \[
            |T_n| \leq t_{n-1}(0.025)   
        \]
        又
        \[
            t_{n-1} (0.005) > t_{n-1} (0.025)
            \ \Rightarrow\ 
            |T_n| < t_{n-1} (0.005)  
        \]
        即此时$\alpha = 0.01$时仍接受$H_0$。
    \end{enumerate}
    综上,当$\alpha = 0.01$时仍接受$H_0$,故选择$A$.
\end{proof}


\end{document}