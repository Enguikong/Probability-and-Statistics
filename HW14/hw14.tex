\documentclass{article}
\usepackage{xeCJK,amsmath,geometry,float,graphicx,amssymb,zhnumber,booktabs,setspace,tasks,verbatim,amsthm,amsfonts,mathdots}
\usepackage{listings,xcolor,diagbox}
\usepackage{unicode-math}
\usepackage{media9}
\usepackage{multimedia}
\geometry{a4paper,scale=0.8}  
\renewcommand\arraystretch{2}
\title{概统作业 (Week 14)}
\author{PB20000113孔浩宇}
\begin{document}
\maketitle
\section{(P306 T6)}  %1
\begin{enumerate}
    \item [(1)]对$X_{(n)}$有分布函数和密度函数
    \[
        F_{n} (x) = 
        \begin{cases}
            0 , & (x \leq 0)\\
            \\
            \displaystyle{\frac{x^n}{\theta^n}} , & (x\in (0,\theta))\\
            \\
            1 . & (x\geq \theta)
        \end{cases} 
        \qquad
        f_{n} (x) = 
        \begin{cases}
            \displaystyle{\frac{n\cdot x^{n-1}}{\theta^{n}}} , & (x\in (0 , \theta))\\
            \\
            0. & (\mbox{其他})
        \end{cases}    
    \]
    故功效函数为
    \[
        \beta_{\Psi }(\theta)
        = \mathbb{P}_{\theta} \left( X_{(n)} \leq 2.5 \big\vert H_{0} \right)  
        = F_{n} (2.5) 
        = {\left(\frac{2.5}{\theta}\right)}^{n}
        \quad (\theta \geq 3)
    \]
    \item [(2)]
\end{enumerate}

\section{(P307 T9)}  %2


\section{}  %3
\begin{enumerate}
    \item [(1)]
    \item [(2)]
\end{enumerate}

\section{(P312 T38)}  %4



\section{}  %5


\end{document}