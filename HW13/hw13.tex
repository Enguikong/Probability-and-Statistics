\documentclass{article}
\usepackage{xeCJK,amsmath,geometry,float,graphicx,amssymb,zhnumber,booktabs,setspace,tasks,verbatim,amsthm,amsfonts,mathdots}
\usepackage{listings,xcolor,diagbox}
\usepackage{unicode-math}
\usepackage{media9}
\usepackage{multimedia}
\geometry{a4paper,scale=0.8}  
\renewcommand\arraystretch{2}
\title{概统作业 (Week 13)}
\author{PB20000113孔浩宇}
\begin{document}
\maketitle
\section{(P234 T23)}  %1
\begin{enumerate}
    \item [(1)]
    \[
        E(\widehat{\theta}_1)    
        = E(\overline{X} + a_n)
        = E(\overline{X}) + E(a_n)
        = E(X) + a_n
    \]
    又
    \[
        E(X)
        = \int_{-\infty}^{+\infty} x f(x ; \theta) dx
        = \int_{\theta}^{+\infty} x\cdot e^{-(x-\theta)}
        = (-x-1) e^{-(x-\theta)} {\big|}_{\theta}^{+\infty}
        = \theta + 1.
    \]
    可以得到
    \[
        E(\widehat{\theta}_1)    
        = \theta + 1 + a_n
        = \theta
        \ \Rightarrow\ 
        a_n = -1 .
    \]
    有$x$的分布函数
    \[
        F(x)
        = \int_{-\infty}^{x} f(x,\theta) dx
        = 
        \begin{cases}
            \ 0, & (x < \theta) \\
            \\
            \ \displaystyle{\int_{\theta}^{x} e^{-(x - \theta)} dx} & (x \geq \theta)
        \end{cases}  
        = 
        \begin{cases}
            \ 0, & (x < \theta) \\
            \\
            \ 1 - e^{-(x - \theta)} & (x \geq \theta)
        \end{cases} 
    \]
    记$X_{(1)} = \min \{ X_1, X_2, \ldots, X_n \}$,则有$x < \theta$时,$X_{(1)}$概率密度$f_{1} (x) = 0$,$x \geq \theta$时,有
    \[
        f_{1} (x)
        = n {\left[ 1 - F(X) \right]}^{n-1} f(x)
        = n e^{- n (x - \theta)} 
    \]
    即
    \[
        f_{1} (x) = 
        \begin{cases}
            \ 0, & (x < \theta) \\
            \\
            \ n e^{-n(x-\theta)}. & (x\geq \theta)
        \end{cases}    
    \]
    有
    \[
        E(\widehat{\theta}_2)
        % = E(X_{(1)} + b_n)
        = E(X_{(1)}) + E(b_n)    
        = \int_{-\infty}^{+\infty} x f_{1}(x) dx + b_n
        = \left(-x - \frac{1}{n}\right) e^{-n(x-\theta)} {\big|}_{\theta}^{+\infty} + b_n
        = \theta + \frac{1}{n} + b_n 
    \]
    可以得到
    \[
        E(\widehat{\theta}_2)
        = \theta + \frac{1}{n} + b_n    
        = \theta
        \ \Rightarrow\ 
        b_n = - \frac{1}{n}
    \]
    即
    \[
        a_n = -1,\quad 
        b_n = -\frac{1}{n}.    
    \]
    \item [(2)]
    \[
        Var(\widehat{\theta}_1)
        = Var\left(\overline{X}-1\right)
        = \frac{1}{n} Var(X)
        = \frac{1}{n} {\left[ E(X^2) - {E(X)}^2 \right]}     
    \]
    又
    \[
        E(X^2)    
        = \int_{-\infty}^{+\infty} x^2 f(x;\theta) dx
        = \int_{\theta}^{+\infty} x^2 \cdot e^{-(x-\theta)} dx
        = \theta^2 + 2\theta + 2
    \]
    故
    \[
        Var(\widehat{\theta}_1)
        = \frac{1}{n}\left[(\theta^2 + 2\theta + 2) - {(\theta + 1)}^2 \right]   
        = \frac{1}{n} .
    \]
    由方差的性质有
    \[
        Var(\widehat{\theta}_2)
        = Var\left(X_{(1)} - \frac{1}{n} \right)
        = Var\left(X_{(1)}\right)
        = E\left({X_{(1)}}^2 \right) - {\left[E\left(X_{(1)}\right) \right]}^2
    \]
    又
    \[
        E\left({X_{(1)}}^2 \right)
        = \int_{-\infty}^{+\infty} x^2 f_{1} (x) dx
        = \int_{\theta}^{+\infty} x^2 \cdot n e^{-n(x-\theta)} dx
        = \theta^2 + \frac{2}{n} \theta + \frac{2}{n^2}
    \]
    故
    \[
        Var(\widehat{\theta}_2)
        = \left(\theta^2 + \frac{2}{n} \theta + \frac{2}{n^2} \right)  - {\left(\theta + \frac{1}{n}\right)}^2
        = \frac{1}{n^2}
    \]
    有$Var(\widehat{\theta}_1) \geq Var(\widetilde{\theta}_2)$,当且仅当$n = 1$时取等,故$\widehat{\theta}_2$更有效.
\end{enumerate}


\section{(P236 T39)}  %2
\begin{enumerate}
    \item [(1)]
    有$x$的密度函数
    \[
        f(x;\theta) =
        \begin{cases}
            \displaystyle{\frac{2x}{\theta}} e^{- x^2 /\theta}, & (x\geq 0)\\
            \\
            0. & (\mbox{其他})
        \end{cases}
    \]
    可得
    \[
        E(X)
        = \int_{\infty}^{+\infty} x f(x;\theta) dx 
        = \int_{0}^{+\infty} \frac{2 x^2}{\theta} e^{-\frac{x^2}{\theta}} dx
        = - \int_{0}^{+\infty} x d e^{-\frac{x^2}{\theta}}
        = \int_{0}^{+\infty} e^{-\frac{x^2}{\theta}} dx - \int_{0}^{+\infty} d \left(x e^{-\frac{x^2}{\theta}}\right)
    \]
    又
    \[
        \int_{0}^{+\infty} e^{-\frac{x^2}{\theta}} dx 
        = \sqrt{\theta} \int_{0}^{+\infty} e^{-{\left(\frac{x}{\sqrt{\theta}}\right)}^2} d\left(\frac{x}{\sqrt{\theta}}\right)
        \xrightarrow{x=\sqrt{\theta}t} \sqrt{\theta} \int_{0}^{+\infty} e^{-t^2} dt 
        = \frac{\pi \theta}{2} .
    \]
    \[
        \int_{0}^{+\infty} d \left(x e^{-\frac{x^2}{\theta}}\right)
        \xrightarrow{t = x e^{-\frac{x^2}{\theta}}} \int_{0}^{0} dt 
        = 0.
    \]
    故
    \[
        E(X) = \frac{\pi \theta}{2}.    
    \]
    另
    \[
        E(X^2)
        = \int_{\infty}^{+\infty} x^2 f(x;\theta) dx 
        = \int_{0}^{+\infty} \frac{2 x^3}{\theta} e^{-\frac{x^2}{\theta}} dx
        \xrightarrow{t = \frac{x^2}{\theta}} \theta \int_{0}^{+\infty} t e^{-t} dt 
        = \theta .   
    \]
    \item [(2)]设$x_1, x_2, \ldots, x_n$为样本观测值,似然函数为
    \[
        L(\theta)
        = \prod_{i=1}^{n} f(x_i)
        = 
        \begin{cases}
            \displaystyle{\frac{2^n \prod\limits_{i=1}^{n} x_i }{\theta^n}} \cdot e^{{-\frac{1}{\theta} \sum\limits_{i=1}^{n} {x_i}^2} }, & (x_1, x_2, \ldots, x_n >0)\\
            \\
            0. & (\mbox{其他})
        \end{cases}    
    \]
    当$x_1, x_2, \ldots, x_n > 0$时,有
    \[
        \ln L(\theta)
        = n \ln 2 + \sum\limits_{i=1}^{n} x_i - n\ln \theta  - \frac{1}{\theta} \sum\limits_{i=1}^{n} {x_i}^2   
    \]
    令
    \[
        \frac{\partial \ln L(\theta)}{\partial \theta}
        = -\frac{n}{\theta} + \frac{1}{\theta^2} \sum\limits_{i=1}^{n} {x_i}^2   
        = 0.  
    \]
    得
    \[
        \widehat{\theta} = \frac{1}{n} \sum\limits_{i=1}^{n} {x_i}^2 .
    \]
    又有$\ln L(\theta)$二阶导
    \[
        \frac{\partial^2 \ln L(\theta)}{\partial \theta^2}
        = \frac{1}{\theta^2}\left(n - \frac{2}{\theta}\sum\limits_{i=1}^{n} {x_i}^2 \right)
        \ \xrightarrow{\theta = \widehat{\theta}}\ 
        \frac{\partial^2 \ln L(\theta)}{\partial \theta^2} {\big|}_{\theta = \widehat{\theta}}
        = -\frac{n}{\theta^2}
        < 0.
    \]
    故$\widehat{\theta} = \frac{1}{n} \sum\limits_{i=1}^{n} {x_i}^2$为$\theta$的最大似然估计量.
    \item [(3)]存在,$a = \theta$.已知$\{{X_i}^2\}$是独立同分布的随机变量序列,记
    \[
        \mu = E(X^2) = \theta,\quad
        S_n = \sum\limits_{i=1}^{n} X_i 
    \]
    由大数定律有,对$\forall\ \varepsilon > 0$,
    \[
        \lim\limits_{n\to \infty} \mathbb{P}\left(\left| S_n / n - \mu \right| \geq \varepsilon \right) = 0
        \ \xrightarrow{\widehat{\theta} = S_n / n }\ 
        \lim\limits_{n\to \infty} \mathbb{P}\left(\left\vert \widehat{\theta} - \theta \right\vert \geq \varepsilon \right) = 0.
    \]
    即证对于$a = \theta$有
    \[
        \widehat{\theta} \xrightarrow{P} a.    
    \]
\end{enumerate}


\section{(P238 T57)}  %3
\begin{proof}
    \begin{enumerate}
        \item []
        \item []对于$X$有密度函数及分布函数
        \[
            f(x) = 
            \begin{cases}
                \displaystyle{\frac{1}{\theta}} , & (x\in (0,\theta))\\
                \\
                0. & (\mbox{其他})
            \end{cases} 
            \qquad
            F(x) = 
            \begin{cases}
                0 , & (x \leq 0)\\
                \\
                \displaystyle{\frac{x}{\theta}} , & (x\in (0,\theta))\\
                \\
                1 . & (x\geq \theta)
            \end{cases}   
        \]
        有$\max \{X_1, X_2, \ldots, X_n\}$的分布函数及密度函数
        \[
            F_{\max} (x) = 
            \begin{cases}
                0 , & (x \leq 0)\\
                \\
                \displaystyle{\frac{x^n}{\theta^n}} , & (x\in (0,\theta))\\
                \\
                1 . & (x\geq \theta)
            \end{cases} 
            \qquad
            f_{\max} (x) = 
            \begin{cases}
                \displaystyle{\frac{n\cdot x^{n-1}}{\theta^{n}}} , & (x\in (0 , \theta))\\
                \\
                0. & (\mbox{其他})
            \end{cases}    
        \]
        有
        \[
            E(\widehat{\theta})
            = \int_{-\infty}^{+\infty} x\cdot f_{\max} (x) dx
            = \int_{0}^{\theta} \frac{n\cdot x^n}{\theta^n} dx
            = \frac{n\cdot \theta}{n + 1}
        \]
        \[
            \lim\limits_{n\to +\infty} E(\widehat{\theta})
            = \lim\limits_{n\to +\infty} \frac{n\cdot \theta}{n + 1}
            = \theta.
        \]
        故$\widehat{\theta}$为$\theta$的相合估计量.又
        \[
            E(\widehat{\theta}) \neq \theta   
        \]
        故$\widehat{\theta}$不是$\theta$的无偏估计量.
    \end{enumerate}
\end{proof}


\section{(P263 T16)}  %4
\begin{enumerate}
    \item [(1)]记样本数据分别为$x_1, x_2, \ldots, x_{10}$,有
    \[
        \sum\limits_{i=1}^{10} {\left(x_i - \mu\right)}^2 = 2.9
    \]
    $1 - \alpha = 0.95,\ \alpha = 0.05$,有
    \[
        \chi_{0.025}^{2} (10) = 3.2470 , \quad
        \chi_{0.975}^{2} (10) = 20.4832
    \]
    取枢轴量
    \[
        Q = \sum\limits_{i=1}^{10} {\left( \frac{X_i - \mu}{\sigma} \right)}^2
        \sim \chi^2_{10} 
    \]
    由
    \[
        P\left(\chi^2_{1 - \alpha / 2} (10) \leq \frac{\sum\limits_{i=1}^{n} {\left(X_i - \mu\right)}^2 }{\sigma^2} \leq \chi^2_{\alpha / 2} (10) \right)    
        = 1 -\alpha
    \]
    有置信区间
    \[
        \left[
            \frac{\sum\limits_{i=1}^{10} {\left(x_i - \mu\right)}^2}{\chi_{0.975}^{2} (10)} 
            \ ,\ 
            \frac{\sum\limits_{i=1}^{10} {\left(x_i - \mu\right)}^2}{\chi_{0.025}^{2} (10)} 
        \right]
        = \left[\frac{2.9}{20.4832}\ ,\ \frac{2.9}{3.2470} \right]
        = [0.1416 , 0.8931]
    \]
    \item [(2)]
    \[
        \overline{x} 
        = \frac{49.5 + 50.4 + 49.7 + 51.1 + 49.4 + 49.7 + 50.8 + 49.9 + 50.3 + 50.0}{10}    
        = 50.08
    \]
    \[
        (n-1) s^2 = \sum\limits_{i=1}^{10} {(x_i - \overline{x})}^2 = 2.8360, \quad
        \chi_{0.025}^{2} (9) = 2.7004, \quad
        \chi_{0.975}^{2} (9) = 19.0228
    \] 
    选取
    \[
        K = \frac{(n-1) S^2}{\sigma^2} \sim \chi^2_{n-1}    
    \]
    由
    \[
        P\left(
            \frac{(n-1)S^2}{\chi^2_{\alpha / 2} (n-1)}       
            \leq \sigma^2 \leq 
            \frac{(n-1)S^2}{\chi^2_{1 - \alpha / 2} (n-1)}  
        \right)    
        = 1 - \alpha
    \]
    有置信区间
    \[
        \left[
            \frac{(n-1) s^2}{\chi_{0.975}^{2} (9)}
            \ ,\ 
            \frac{(n-1) s^2}{\chi_{0.975}^{2} (9)}
        \right]
        = \left[\frac{2.8360}{19.0228}\ ,\ \frac{2.8360}{2.7004}\right]
        = [0.1491 , 1.0502]
    \]
\end{enumerate}


\section{(P263 T19)}  %5
记$Y = \displaystyle{\frac{X}{\theta}},\ Y_{i} = \displaystyle{\frac{X_i}{\theta}},\ X_{(n)} = \max\{X_1, X_2, \ldots, X_n \},\ Y_{n} = \max\{Y_1, Y_2, \ldots, Y_n\}$,则有
\[
    Y \sim U(0,1),\quad
    X_{(n)} = \theta\cdot Y_{(n)},\quad
    P\left(X_{(n)} \leq \theta \leq c_n X_{(n)}\right)
    = P\left({c_n}^{-1} \leq Y_{(n)} \leq 1\right)
\]
有$Y_{(n)}$的分布函数及密度函数
\[
    F_{\max} (y) = 
    \begin{cases}
        0 , & (y \leq 0)\\
        \\
        y^n , & (y\in (0, 1))\\
        \\
        1 . & (y\geq 1)
    \end{cases} 
    \qquad
    f_{\max} (y) = 
    \begin{cases}
        n\cdot y^{n-1} , & (y\in (0 , 1))\\
        \\
        0. & (\mbox{其他})
    \end{cases}    
\]
若有$c_n$使得$\left[X_{(n)} , c_n X_{(n)} \right]$为$\theta$的$1-\alpha$置信系数,则有$\forall\ \theta \in \Theta$
\[
    P\left(X_{(n)} \leq \theta \leq c_n X_{(n)}\right) = 1 - \alpha
    \ \Leftrightarrow\ 
    P\left({c_n}^{-1} \leq Y_{(n)} \leq 1\right) = 1 - F_{\max} ({c_n}^{-1}) = 1 - \alpha
\]
又显然$0\leq c_n \leq 1$,有
\[
    1 - F_{\max} ({c_n}^{-1})
    = 1 - {c_n}_{-n}
    = 1 - \alpha    
    \ \Rightarrow\ 
    c_n = \alpha^{-\frac{1}{n}}
\]
即存在$c_n = \alpha^{-\frac{1}{n}}$满足要求.
\end{document}