\documentclass{article}
\usepackage{xeCJK,amsmath,geometry,float,graphicx,amssymb,zhnumber,booktabs,setspace,tasks,verbatim,amsthm,amsfonts,mathdots}
\usepackage{listings,xcolor,diagbox}
\geometry{a4paper,scale=0.8}  
\renewcommand\arraystretch{2}
\title{概统作业 (Week 2)}
\author{PB20000113孔浩宇}
\begin{document}
\maketitle
\section{}  %1
\begin{enumerate}
    \item []设挑出第一箱为事件A,挑出第二箱为事件B
    \item [(1)]
    \item [(2)]
\end{enumerate}

\section{}  %2
\begin{enumerate}
    \item []设方程有实根为事件$M$,方程有重根为事件$N$,$B=i$记为$B_i$,$C=i$记为$C_i$.
    \item [(1)]$M\Leftrightarrow B^2-4C\geq 0$.
    \begin{align*}
        P(M)
        & = 
        P(B_2 C_1) + P(B_3 (C_1+C_2)) + P(B_4 (C_1 + C_2 + C_3 + C_4))+P(B_5)+P(B_6)\\
        & =
        \frac{1}{6}\times \frac{1}{6} + \frac{1}{6}\times \frac{2}{6} +
        \frac{1}{6}\times \frac{4}{6} + \frac{1}{6} + \frac{1}{6} \\
        & =
        \frac{1}{36} + \frac{2}{36} + \frac{4}{36} + \frac{1}{3}\\
        & =
        \frac{19}{36}.
    \end{align*}
    \item [(2)]$N\Leftrightarrow B^2 = 4C$.
    \begin{align*}
        P(N)
        & =
        P(B_2 C_1) + P(B_4 C_4)\\
        & =
        \frac{1}{6}\times\frac{1}{6} + \frac{1}{6}\times \frac{1}{6}\\
        & =
        \frac{1}{36} + \frac{1}{36}\\
        & =
        \frac{1}{18}.
    \end{align*}
\end{enumerate}

\section{}  %3
\begin{proof}
    \begin{align*}
        P(A|B)=P(A|B^{C})
        \Leftrightarrow &
        \frac{P(AB)}{P(B)} = \frac{P(AB^C)}{P(B^{C})}\\
        \Leftrightarrow &
        P(AB)\cdot P(B^{C}) = P(AB^{C})\cdot P(B)\\
        \Leftrightarrow &
        P(AB)\cdot (1-P(B)) = P(AB^{C})\cdot P(B)\\
        \Leftrightarrow &
        P(AB) = P(B)\cdot \left[P(AB)+P(AB^{C})\right]\\
        \Leftrightarrow &
        P(AB) = P(B)\cdot P(A)\cdot \left[P(B)+P(B^{C})\right]\\
        \Leftrightarrow &
        P(AB) = P(B)\cdot P(A).
    \end{align*}
\end{proof}

\section{}  %4
\begin{align*}
    P(A\cup B\cup C)
    & = 
    P(A)+P(B)+P(C)-\left(P(AB)+P(BC)+P(AC)\right)+P(AC\cap B)\\
    & =
    1-(\frac{1}{8}+\frac{1}{8}+0)+0\\
    & =
    \frac{3}{4}.
\end{align*}

\section{}  %5



\end{document}