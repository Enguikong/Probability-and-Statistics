\documentclass{article}
\usepackage{xeCJK,amsmath,geometry,float,graphicx,amssymb,zhnumber,booktabs,setspace,tasks,verbatim,amsthm,amsfonts,mathdots}
\usepackage{listings,xcolor,diagbox}
\usepackage{unicode-math}
\usepackage{media9}
\usepackage{multimedia}
\geometry{a4paper,scale=0.8}  
\renewcommand\arraystretch{2}
\title{概统作业 (Week 12)}
\author{PB20000113孔浩宇}
\begin{document}
\maketitle
\section{}  %1
\subsection*{选B}
\begin{enumerate}
    \item [(A)]
    \[
        (X_i - \mu) \sim N(0,1)
        \ \Rightarrow\ 
        \sum\limits_{i=1}^{n} {(X_i - \mu)}^2 \sim \chi _{n}^{2}
    \]
    \item [(B)]
    \[
        (X_n - X_1) \sim N(0,2)
        \ \Rightarrow\ 
        \frac{{(X_n - X_1)}^2}{2} \sim \chi_{1}^{2}
        \ \Rightarrow\ 
        2 {(X_n - X_1)}^2 \mbox{不服从} \chi^2 \mbox{分布}
    \]
    \item [(C)]
    \[
        \sum\limits_{i=1}^{n} {(X_n - \overline{X})}^2
        = (n-1) S^2 \sim \chi_{n-1}^{2}
    \]
    \item [(D)]
    \[
        \sqrt{n} (\overline{X} - \mu) \sim N(0,1)
        \ \Rightarrow\ 
        n {(\overline{X} - \mu)}^2 \sim \chi_{1}^{2}    
    \]
\end{enumerate}

\section{(P205 T15)}  %2
\[
    \begin{cases}
        \ E(X_1 - 2 X_2) = 0, \quad & Var(X_1 - 2 X_2) = 20\\
        \ E(3 X_3 - 4 X_4) = 0, \quad & Var(3 X_3 - 4 X_4) = 100
    \end{cases}
    \ \Rightarrow\ 
    \begin{cases}
        \ X_1 - 2 X_2 & \sim N(0, 20)\\
        \ 3 X_3 - 4X_4 & \sim N(0, 100)
    \end{cases}
\]
要使$T$服从$\chi^2$分布,有
\[
    \begin{cases}
        \sqrt{a} (X_1 - 2 X_2)  & \sim N(0, 1)\\
        \sqrt{b} (3X_3 - 4X_4)  & \sim N(0, 1)
    \end{cases}
    \ \Rightarrow\ 
    20a = 100b = 1
    \ \Rightarrow\ 
    a = \frac{1}{20},\ b =\frac{1}{100}
\]

\section{(P205 T16)}  %3
\begin{proof}
    \begin{enumerate}
        \item []
        \item []服从自由度为$2$的$t$分布.不妨设$X_i \sim N(\mu, \sigma^2)\ (1\leq i\leq 9)$,有
        \[
            E(Y_1) = E(Y_2) = \mu,\quad
            Var(Y_1) = \frac{\sigma^2}{6},\quad
            Var(Y_2) = \frac{\sigma^2}{3}
        \]
        故
        \[
            E(Y_1 - Y_2) = 0,\quad
            Var(Y_1 - Y_2) = \frac{\sigma^2}{2},\quad
            Y_1 - Y_2 \sim N(0, \frac{\sigma^2}{2})  
        \]
        记$M = \frac{\sqrt{2}}{\sigma} (Y_1 - Y_2)$,则
        \[
            M \sim N(0,1)    
        \]
        记$N = \frac{2 S^2}{\sigma^2}$,由定理可得
        \[
            N = \frac{2 S^2}{\sigma^2} \sim \chi_{2}^2
        \]
        则
        \[
            Z = \frac{\sqrt{2} (Y_1 - Y_2)}{S} = \frac{M}{\sqrt{N/2}} \sim t_2 .
        \]
    \end{enumerate}
\end{proof}


\section{(P232 T8)}  %4
\begin{enumerate}
    \item [(1)]
    \[
        E(X) 
        = \int_{-\infty}^{+\infty} x \cdot f(x) dx   
        = \int_{0}^{\theta} \frac{x}{2\theta} dx + \int_{\theta}^{1} \frac{x}{2(1-\theta)} dx
        = \frac{1}{4} + \frac{\theta}{2}
    \]
    有
    \[
        \overline{X} = \frac{1}{4} + \frac{\widehat{\theta} }{2} 
        \ \Rightarrow\ 
        \widehat{\theta} = 2 \overline{X} - \frac{1}{2} .
    \]
    \item [(2)]不是。理由如下
    \[
        E(4 \overline{X}^2) 
        = 4 \left[ Var(\overline{X}) + {\left( E(\overline{X}) \right)}^2 \right] 
        = 4 \left[ \frac{Var(X)}{n} + {\left( E(X) \right)}^2 \right]
    \]
    又
    \[
        E(X) = \frac{1}{4} + \frac{\theta}{2},\quad
        E(X^2) = \int_{-\infty}^{+\infty} x^2 f(x) dx = \frac{2\theta^2 + \theta + 1}{6}  
    \]
    故
    \[
        Var(X) 
        = E(X^2) - {E(X)}^2
        = \frac{2\theta^2 + \theta + 1}{6} - {\left(\frac{1}{4} + \frac{\theta}{2} \right)}^2
        = \frac{\theta^2}{12} - \frac{\theta}{12} + \frac{5}{48}.
    \]
    原式即为
    \[
        E(4 \overline{X}^2) 
        = 4 \left[\left( \frac{\theta^2}{12n} - \frac{\theta}{12n} + \frac{5}{48n} \right) + \left( \frac{\theta^2}{4} + \frac{\theta}{4} + \frac{1}{16} \right)\right]
        = \frac{3n + 1}{3n} \theta^2 + \frac{3n-1}{3n} \theta + \frac{3n+5}{12}
        \neq \theta^2
    \]
    即证$4 \overline{X}^2$不是$\theta^2$的无偏估计量.
\end{enumerate}

\section{(P234 T27)}  %5
有最大似然函数
\[
    L(X_1,X_2,\ldots,X_n ; \lambda)
    = \prod_{i=1}^{n} \frac{\lambda^{x_i}}{X_i !} e^{-\lambda}
    = e^{-n\lambda} \prod_{i=1}^{n} \frac{\lambda^{X_i}}{X_i !}.
\]
\[
    \ln L 
    = -n\lambda + \sum\limits_{i=1}^{n} \left(X_i \ln \lambda - \ln X! \right)    
\]
令
\[
    \frac{\partial \ln L}{\partial \lambda} 
    = \frac{1}{\lambda} \sum\limits_{i=1}^{n} X_i - n 
    = 0. 
\]
得
\[
    \widehat{\lambda} = \frac{1}{n}\sum\limits_{i=1}^{n} X_i = \overline{X}  
\]
检验二阶导
\[
    \frac{\partial^2 \ln L}{\partial \lambda^2} 
    = - \frac{1}{\lambda^2} \sum\limits_{i=1}^{n} X_i 
\]
对于所有$\lambda$与$\overline{X}$非零时为负,故所求驻点为$L$极大值点,$P(X=0)$的最大似然估计为
\[
    P(X = 0) = e^{-\widehat{\lambda}} = e^{- \overline{X}}
\]

\section{(P235 T29)}    %6
\begin{enumerate}
    \item [(1)]即$X \sim U(\theta, 0),\ \theta \in \Theta$,有矩估计
    \[
        E(X)
        = \frac{0 + \theta}{2}
        = \overline{X}
        \ \Rightarrow\ 
        \widehat{\theta}
        = 2 \overline{X}    
    \]
    有最大似然函数
    \[
        L(X_1,X_2,\ldots,X_n ; \theta)
        = \prod_{i=1}^{n} P(X = X_i)
        = 
        \begin{cases}
            \displaystyle{\frac{1}{{(-\theta)}^{n}}}, & X_1, X_2, \ldots, X_n \in [\theta, 0]\\
            \\
            0. & \mbox{其他}
        \end{cases}
    \]
    显然当${(-\theta)}^{n}$最小,即$\theta$最大时,$L$最大,又
    \[
        \theta \leq \min\{X_1, X_2, \ldots, X_n\},\ 
        \theta \in (-\infty, 0)
    \]
    有最大似然估计
    \[
        \widehat{\theta} = \min\{X_1, X_2, \ldots, X_n\} .
    \]
    \item [(2)]即$X \sim U(\theta, 2\theta),\ \theta \in \Theta$,有矩估计
    \[
        E(X)
        = \frac{\theta + 2\theta}{2}
        = \overline{X}
        \ \Rightarrow\ 
        \widehat{\theta}
        = \frac{2}{3} \overline{X}    
    \]
    有最大似然函数
    \[
        L(X_1,X_2,\ldots,X_n ; \theta)
        = \prod_{i=1}^{n} P(X = X_i)
        = 
        \begin{cases}
            \displaystyle{\frac{1}{\theta^{n}}}, & X_1, X_2, \ldots, X_n \in [\theta, 2\theta]\\
            \\
            0. & \mbox{其他}
        \end{cases}
    \]
    显然当${\theta}^{n}$最小时,$L$最大,又
    \[
        \theta \leq \min\{X_1, X_2, \ldots, X_n\},\ 
        \theta \geq \max\{X_1, X_2, \ldots, X_n\},\ 
        \theta \in (0, +\infty)
    \]
    有最大似然估计
    \[
        \widehat{\theta} =\frac{1}{2} \max\{X_1, X_2, \ldots, X_n\} .
    \]
\end{enumerate}


\end{document}