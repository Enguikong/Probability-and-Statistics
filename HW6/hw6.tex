\documentclass{article}
\usepackage{xeCJK,amsmath,geometry,float,graphicx,amssymb,zhnumber,booktabs,setspace,tasks,verbatim,amsthm,amsfonts,mathdots}
\usepackage{listings,xcolor,diagbox}
\geometry{a4paper,scale=0.8}  
\renewcommand\arraystretch{2}
\title{概统作业 (Week 6)}
\author{PB20000113孔浩宇}
\begin{document}
\maketitle
\section{(P116 T6)}  %1
\begin{enumerate}
    \item [(1)]由题意可得$i,j \in \mathbb{N}^{*},\ 1\leq i < j $,有
    \[
        P(X=i, Y=j)=
        \begin{cases}
            p^2 {(1-p)}^{j-2}   \quad &(1\leq i < j)\\
            \\
            0   &\mbox{其他}
        \end{cases}.
    \]
    \item [(2)]
    \[
        P(X=i) = p {(1-p)}^{i-1}.
        \quad (i\in \mathbb{N}^{*})
    \]
    \[
        P(Y=j) = (j-1) \cdot p^{2} {(1-p)}^{j-2} .  
        \quad (j\in \mathbb{N}^{*},\ j\geq 2)
    \]
\end{enumerate}

\section{(P116 T10)}  %2
\begin{enumerate}
    \item [(1)]
    \[
        F(x,y) =  \int_{-\infty}^{x} du \int_{-\infty}^{y} dv \cos{u} \cos{v}
        = \begin{cases}
            \ 0,    & (x\leq 0\mbox{或}\ y\leq 0)\\
            \\
            \ \sin{x} \sin{y},    & (0<x<\displaystyle{\frac{\pi}{2}},\ 0<y<\displaystyle{\frac{\pi}{2}}) \\
            \\
            \ \sin{x},    & (0<x<\displaystyle{\frac{\pi}{2}},\ y\geq \displaystyle{\frac{\pi}{2}})\\
            \\
            \ \sin{y},    & (x\geq \displaystyle{\frac{\pi}{2}},\ 0<y<\displaystyle{\frac{\pi}{2}})\\
            \\
            \ 1.  & (x\geq \displaystyle{\frac{\pi}{2}},\ y\geq \displaystyle{\frac{\pi}{2}})
        \end{cases}
    \]
    \item [(2)]
    \[
        \begin{aligned}
            P(0< X <\pi /4,\ \pi /4 < Y < \pi /2)
            =&
            F(\pi /4, \pi /2) - F(\pi /4, \pi/4) - F(0, \pi /2) + F(0, \pi /4)\\
            =&  
            \displaystyle{\frac{\sqrt{2}}{2} - \frac{1}{2} - 0 + 0}\\
            =&
            \displaystyle{\frac{\sqrt{2}-1}{2}}.
        \end{aligned}
    \]
\end{enumerate}

\section{(P116 T9)}  %3
\begin{enumerate}
    \item [(1)]
    \[
        \begin{cases}
            \ F(+\infty, +\infty) & = a(b+ \pi /2) (c+ \pi /2) = 1\\
            \ F(-\infty, -\infty) & = a(b- \pi /2) (c- \pi /2) = 0\\
            \ F(0, -\infty) & = ab (c- \pi /2) = 0\\
            \ F(-\infty, 0) & = ac (b- \pi /2) = 0
        \end{cases}   
        \ \Rightarrow\ 
            a = \displaystyle{\frac{1}{\pi ^ 2}},\ 
            b = \displaystyle{\frac{\pi}{2}},\ 
            c = \displaystyle{\frac{\pi}{2}}
    \]
    \item [(2)]
    \[
        \begin{aligned}
            P(X>0, Y>0)
            & =
            1 - P(X\leq 0) -P(Y\leq 0) + P(X\leq 0, Y\leq 0)\\
            & = 
            1 - F(0, +\infty) - F(+\infty, 0) + F(0,0)\\
            & =
            1 - \frac{1}{2} - \frac{1}{2} + \frac{1}{4}\\
            & =  
            \frac{1}{4}        
        \end{aligned}
    \]
\end{enumerate}

\section{(P116 T5)}  %4
    由题意可得
    \[
        \begin{aligned}
            &P(X=-1, X+Y=0) = P(X=-1)\cdot P(X+Y=0)\\ 
            \ \Leftrightarrow&
            P(X=-1, Y=1) = P(X=-1) \cdot [P(X=1,Y=-1) + P(X=-1,Y=1)]\\
            \ \Leftrightarrow&
            a = (a+0.2) \cdot (a+b)
        \end{aligned}
    \]
    又$a+b+0.2+0.3 = 1$,有
    \[
        \begin{cases}
            a+b = 0.5 \\
            a = (a+0.2) (a+b)
        \end{cases}   
        \ \Rightarrow\ 
        a= 0.2,\ b=0.3.
    \]

\section{(P117 T17)}  %5
\begin{enumerate}
    \item [(1)]对于$X=x(0<x<1)$,有
    \[
        f_{Y|X}(y|x) =  
        \displaystyle{\frac{f(x,y)}{f_1(x)}}=
        \begin{cases}
            \displaystyle{\frac{3y^2}{x^3}},     & (0<y<x)\\
            \\
            0.   &(\mbox{其他})
        \end{cases} 
        \ \Rightarrow\ 
        f(x,y)= f_{Y|X}(y|x) \cdot f_1(x) =  
        \begin{cases}
            \displaystyle{\frac{9y^2}{x}} & (0<y<x<1)\\
            \\
            0.  & (\mbox{其他})
        \end{cases}
    \]
    又$\forall\ x\notin (0,1),\ f_{X}(x) = 0 \Rightarrow f(x,y)=0$,故
    \[
        f(x,y)=   
        \begin{cases}
            \displaystyle{\frac{9y^2}{x}} & (0<y<x<1)\\
            \\
            0.  & (\mbox{其他})
        \end{cases}
    \]
    \item [(2)]
    \[
        f_{Y} (y)=
        \int_{\mathbb{R}} f(x,y) dx =
        \int_{y}^{1} f(x,y) dx = 
        \begin{cases}
            \ -9y^2 \ln{y},   &(0<y<1)\\
            \ 0.  &\mbox{其他}    
        \end{cases}
    \]
\end{enumerate}

\end{document}