\documentclass{article}
\usepackage{xeCJK,amsmath,geometry,float,graphicx,amssymb,zhnumber,booktabs,setspace,tasks,verbatim,amsthm,amsfonts,mathdots}
\usepackage{listings,xcolor,diagbox}
\usepackage{unicode-math}
\usepackage{media9}
\usepackage{multimedia}
\geometry{a4paper,scale=0.8}  
\renewcommand\arraystretch{2}
\title{概统2023}
\author{PB20000113孔浩宇}
\begin{document}
\maketitle
\subsection*{}
设两个小孩分别为甲乙,有一个春天出生的男孩是事件A,两个小孩都是男孩为事件B。
每个小孩的可能有春/夏/秋/冬,男/女,共8种情况,2个小孩共有$8\times 8 = 64$种情况。
已知有一个春男(事件A),则有$7\times 7 = 49$种情况不会发生,此时只有$64 - 49 = 15$个基本事件。
对于事件B,有甲:春男\& 乙:夏秋冬男,乙:春男\& 甲:夏秋冬男, 甲乙都是春男 共7种情况
\[
    P(B|A)
    = \frac{P(AB)}{P(A)}
    = \frac{7}{15}
\]
\end{document}