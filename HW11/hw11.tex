\documentclass{article}
\usepackage{xeCJK,amsmath,geometry,float,graphicx,amssymb,zhnumber,booktabs,setspace,tasks,verbatim,amsthm,amsfonts,mathdots}
\usepackage{listings,xcolor,diagbox}
\usepackage{unicode-math}
\geometry{a4paper,scale=0.8}  
\renewcommand\arraystretch{2}
\title{概统作业 (Week 11)}
\author{PB20000113孔浩宇}
\begin{document}
\maketitle
\section{(P176 T42)}  %1
\begin{proof}
    \begin{enumerate}
        \item []
        \item [(1)]
        \begin{align*}
            X_n \xrightarrow{P} X     
            &\Rightarrow\ 
            \forall\ \frac{\varepsilon}{2} >0,\ \lim\limits_{n \to \infty} P(|X_n - X|\geq \frac{\varepsilon}{2}) = 0.\\
            &\Rightarrow\ 
            \forall\ \frac{\varepsilon}{2} >0,\ \forall\ \frac{\delta}{2} >0 ,\ \exists\ N_1 >0,\ \forall\ n>N_1,\ P(|X_n - X|\geq \frac{\varepsilon}{2}) < \frac{\delta}{2}.
        \end{align*}
        \item [(2)]
        \begin{align*}
            Y_n \xrightarrow{P} Y     
            &\Rightarrow\ 
            \forall\ \frac{\varepsilon}{2} >0,\ \lim\limits_{n \to \infty} P(|Y_n - Y|\geq \frac{\varepsilon}{2}) = 0.\\
            &\Rightarrow\ 
            \forall\ \frac{\varepsilon}{2} >0,\ \forall\ \frac{\delta}{2} >0 ,\ \exists\ N_1 >0,\ \forall\ n>N_1,\ P(|Y_n - Y|\geq \frac{\varepsilon}{2}) <  \frac{\delta}{2}.
        \end{align*}
        \item [(3)]$\forall\ \varepsilon >0,\ \forall\ \delta >0,\ \exists\ N_1 , N_2 >0$,
        \[
            \forall\ n>\max\{N_1 , N_2\},\ 
            P(|X_n + Y_n - X - Y|\geq \varepsilon) 
            \leq
            P(|X_n - X|\geq \frac{\varepsilon}{2}) + P(|Y_n - Y|\geq \frac{\varepsilon}{2})  
            < \delta.  
        \]
        即证$\forall\ \varepsilon >0$,有
        \[
            \lim\limits_{n \to \infty} P(|X_n + Y_n - X - Y|\geq \varepsilon) = 0.
        \]
        即证
        \[
            X_n + Y_n \xrightarrow{P} X + Y .     
        \]
    \end{enumerate}
\end{proof}

\section{(P177 T45)}  %2
\begin{enumerate}
    \item [(1)]记500次独立重复试验中,事件A发生的次数为$X$,则$X\sim B(500,0.2)$
    \[
        E(X) = 500 \times 0.2 = 100. 
        \qquad 
        Var(X) = 500\times 0.2 \times (1 - 0.2) = 80.
    \]
    \[
        P(80<X<120) 
        = P(|X-100|<20) 
        = 1 - P(|X-100|\geq 20) 
        \geq  1 - \frac{80}{20^2} 
        = 0.8 .    
    \]
    即用切比雪夫不等式估计,概率大约为$0.8$.
    \item [(2)]记第$i$次实验中事件A发生的次数为$X_i$,则$X_i \sim B(1,0.2)$,有
    \[
        S_n = \sum\limits_{i=1}^{500} X_n ,\quad
        np = 100,\quad
        \sqrt{np(1-p)} = \sqrt{80} = 4\sqrt{5}.
    \]
    记事件A发生次数在80和120之间为事件M,则
    \[
        P(M)
        = P(80 \leq S_n \leq 120)
        = P\left(-\sqrt{5} \leq \frac{S_n - np}{\sqrt{np(1-p)}} \leq \sqrt{5} \right)
        = 2 P\left(\frac{S_n - np}{\sqrt{np(1-p)}} \leq \sqrt{5} \right) - 1
    \]
    又
    \[
        \frac{S_n - np}{\sqrt{np(1-p)}} 
        \xrightarrow{\mathcal{L} }
        N(0,1)    
        \ \Rightarrow\ 
        P\left(\frac{S_n - np}{\sqrt{np(1-p)}} \leq \sqrt{5} \right)
        \approx \Phi(\sqrt{5})
        \approx 0.9868
    \]
    有
    \[
        P(M) \approx 2\times 0.9868 - 1 = 0.9736 .
    \]
    即用中心极限定理估计,概率大约为$0.9736$.
\end{enumerate}


\section{(P175 T51)}  %3
\begin{enumerate}
    \item [(0)]设第$i$件产品的组装时间为$X_i$分钟,则$X_i \sim Exp(\lambda)$,有
    \[
        \mu = E(X_i) = \frac{1}{\lambda} = 10
        \ \Rightarrow\ 
        \lambda = \frac{1}{10},\ 
        \sigma^2 = Var(X_i) = \frac{1}{\lambda^2} = 100.    
    \]
    \item [(1)]记$S_n = \sum\limits_{i=1}^{100} X_i$
    \[
        P(15\times 60 \leq S_n \leq 20\times 60)  
        = P(S_n \leq 1200) - P(S_n \leq 900)  
    \]
    由中心极限定理有
    \[
        P(S_n \leq 1200)
        = P\left(\frac{\sqrt{n}(S_n / n - \mu)}{\sigma} \leq 2 \right)    
        \approx \Phi(2)
    \]
    \[
        P(S_n \leq 900)
        = P\left(\frac{\sqrt{n}(S_n / n - \mu)}{\sigma} \leq -1 \right)    
        \approx \Phi(-1)
    \]
    故
    \[
        P(900 \leq S_n \leq 1200)
        \approx \Phi(2) - \Phi(-1)
        = \Phi(2) + \Phi(1) - 1
        \approx 0.8185.
    \]
    \item [(2)]设保证有$95\% $的可能性下,$16h$内最多可以组装$k$件产品,
    记$S_k = \sum\limits_{i=1}^{k} X_i$,则有
    \[
        P\left( S_k \leq 16\times 60\right)
        = P\left( S_k \leq 960 \right)
        \geq 0.95
    \]
    又由中心极限定理
    \[
        P(S_k \leq 960)
        = P\left(\frac{\sqrt{k}(S_k / k - \mu)}{\sigma} \leq \frac{\sqrt{k}(960/ k - \mu)}{\sigma} \right)
        = P\left(\frac{\sqrt{k}(S_k / k - \mu)}{\sigma} \leq \frac{96 - k}{\sqrt{k}} \right)
        \approx \Phi\left( \frac{96 - k}{\sqrt{k}} \right)
    \]
    有
    \[
        \Phi\left( \frac{96 - k}{\sqrt{k}} \right) \geq 0.95,\ 
        \Phi(1.645)=0.950015
        \ \Rightarrow\ 
        \frac{96 - k}{\sqrt{k}} \geq 1.645
        \ \Rightarrow\ 
        k_{\max} = 81 .
    \]
    即要保证有$95\% $的可能性下,$16h$内最多可以组装$81$件产品,
\end{enumerate}

\clearpage
\section{(P175 T59)}  %4
\begin{proof}
    \begin{enumerate}
        \item []
        \item []
        设$X_i\ (i=0,1,2,\ldots)$为服从参数$\lambda = 1$的泊松分布,且相互独立的一系列分布,则
        \[
            \mu = E(X_i) = \lambda = 1,\quad
            \sigma^2 = Var(X_i) = \lambda = 1   
        \]
        记$S_n = \sum\limits_{i=0}^{n} X_i$,由中心极限定理有
        \[
            \lim\limits_{n\to \infty} P\left(\frac{\sqrt{n} (S_n / n - \mu) }{\sigma} \leq 0 \right)
            = \lim\limits_{n\to \infty} P(S_n \leq n)
            = \Phi(0)
            = \frac{1}{2}.
        \]
        即
        \[
            \lim\limits_{n\to \infty} P\left(\sum\limits_{i=0}^{n} X_i \leq n\right)     
            = \frac{1}{2}.
        \]
        又$Y_n = \sum\limits_{i=0}^{n} X_i$服从参数$\lambda = n$的泊松分布,故
        \[
            P\left( \sum\limits_{i=0}^{n} X_i \leq n \right) 
            = P\left( Y_n \leq n \right)   
            = \sum\limits_{i=0}^{n} \frac{e^{-n} \cdot n^{k}}{k!}
            = e^{-n} \sum\limits_{i=0}^{n} \frac{n^k}{k!}
        \]
        即证
        \[
            \lim\limits_{n\to \infty}e^{-n} \sum\limits_{i=0}^{n} \frac{n^k}{k!}
            = \lim\limits_{n\to \infty} P\left(\sum\limits_{i=0}^{n} X_i \leq n\right)     
            = \frac{1}{2}.
        \]
    \end{enumerate}
    
    
\end{proof}


\section{(P240 T8)}  %5
\begin{enumerate}
    \item [(1)]
    样本空间:
    \[
        \Omega 
        = \{(x_1, x_2, \ldots, x_5)\ |\ x_i = 0,1\ \mbox{且} \ 1\leq i \leq 5 \}
    \]
    抽样分布:$记k = \sum\limits_{i=1}^{5} x_i$,
    \[
        P(X_1 = x_1, X_2 = x_2, \ldots, X_5 = x_5)
        = p^{k} {(1-p)}^{5 - k}.
    \]
    \item [(2)]
    统计量:
    \[
        X_1 + X_2,\quad
        \min_{1\leq i \leq 5} X_i
    \]
    非统计量:因为依赖于未知的参数$p$,所以不是统计量
    \[
        X_5+2p,\quad
        X_5 - E(X_1),\quad
        \frac{{(X_5 - X_1)}^2}{Var(X_1)}    
    \]
\end{enumerate}

\end{document}